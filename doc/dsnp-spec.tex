%
%   Copyright 2001-2009 Adrian Thurston <thurston@complang.org>
%

%   This file is part of Ragel.
%
%   Ragel is free software; you can redistribute it and/or modify
%   it under the terms of the GNU General Public License as published by
%   the Free Software Foundation; either version 2 of the License, or
%   (at your option) any later version.
%
%   Ragel is distributed in the hope that it will be useful,
%   but WITHOUT ANY WARRANTY; without even the implied warranty of
%   MERCHANTABILITY or FITNESS FOR A PARTICULAR PURPOSE.  See the
%   GNU General Public License for more details.
%
%   You should have received a copy of the GNU General Public License
%   along with Ragel; if not, write to the Free Software
%   Foundation, Inc., 59 Temple Place, Suite 330, Boston, MA  02111-1307  USA 

% TODO: Need a section on the different strategies for handline recursion.

\documentclass[letterpaper,11pt,oneside]{article}
%\usepackage{graphicx}
\usepackage{comment}
%\usepackage{multicol}
%\usepackage[
%	colorlinks=true,
%	linkcolor=black,
%	citecolor=green,
%	filecolor=black,
%	urlcolor=black]{hyperref}

\topmargin -0.20in
\oddsidemargin 0in
\textwidth 6.5in
\textheight 9in

\setlength{\parskip}{0pt}
\setlength{\topsep}{0pt}
\setlength{\partopsep}{0pt}
\setlength{\itemsep}{0pt}

%% \input{version}
%% 
%% \newcommand{\verbspace}{\vspace{10pt}}
%% \newcommand{\graphspace}{\vspace{10pt}}
%% 
%% \renewcommand\floatpagefraction{.99}
%% \renewcommand\topfraction{.99}
%% \renewcommand\bottomfraction{.99}
%% \renewcommand\textfraction{.01}   
%% \setcounter{totalnumber}{50}
%% \setcounter{topnumber}{50}
%% \setcounter{bottomnumber}{50}
%% 
%% \newenvironment{inline_code}{\def\baselinestretch{1}\vspace{12pt}\small}{}

\begin{document}

%
% Title page
%
\thispagestyle{empty}
\begin{center}
{\huge DSNP: Distributed Social Network Protocol}\\
\vspace*{12pt}
{\Large Specification and Software Documentation}\\
\vspace{12pt}
by\\
\vspace{12pt}
{\large Dr. Adrian D. Thurston}\\
\end{center}

\pagenumbering{roman}

\section*{License}

Copyright \copyright\ 2007-2011 Adrian D. Thurston

\vspace{5pt}

{\bf\it\noindent Permission to use, copy, modify, and/or distribute this software for any
purpose with or without fee is hereby granted, provided that the above
copyright notice and this permission notice appear in all copies.}

\vspace{5pt}

{\bf\it\noindent THE SOFTWARE IS PROVIDED "AS IS" AND THE AUTHOR DISCLAIMS ALL WARRANTIES
WITH REGARD TO THIS SOFTWARE INCLUDING ALL IMPLIED WARRANTIES OF
MERCHANTABILITY AND FITNESS. IN NO EVENT SHALL THE AUTHOR BE LIABLE FOR
ANY SPECIAL, DIRECT, INDIRECT, OR CONSEQUENTIAL DAMAGES OR ANY DAMAGES
WHATSOEVER RESULTING FROM LOSS OF USE, DATA OR PROFITS, WHETHER IN AN
ACTION OF CONTRACT, NEGLIGENCE OR OTHER TORTIOUS ACTION, ARISING OUT OF
OR IN CONNECTION WITH THE USE OR PERFORMANCE OF THIS SOFTWARE.}

\tableofcontents

%
% Chapter 1
%


% Topics:
% RSA Keys allocated to each identity
%
% Broadcast Keys
%
% Properties.
%  Forgery
%  Privacy
%
% Protocol
%  Friendship Initialization

\pagenumbering{arabic}

\section{Introduction}

This is the introduction

\section{Identity Initialization}

A user name and password is requested. These credentials will be used by the
user to login as the owner of the identity.

A URI consiting of the site root name followed by the user name as the last
component is assigned. This is an internet facing web address that the user has
control of (possibly through a service provider). It represents their identity.

An RSA key pair is created. The key has no passphrase. 

The public portion of the key is made publically available. The key must always
be fetched using SSL. This guarantees that a key has been provided by the
server hosting the identity and has not been tampered with.

\section{Owner Login}

User submits credentials via a web form underneath URI and recieves a session
cookie that indicates to the server that the user is logged in.

\section{Friendship Request}

The person requesting friendship is the friender, with identity URI.
The person who is receiving the request is the friendee, with FR-URI

\begin{enumerate}
\item The friender fills in a friend request form at FR-URI
	\begin{enumerate}
	\item friender answers a challenge (generic CAPTCHA, or personalized)
	\item friender submits URI
	\end{enumerate}

\item Friendee server processes the request:
	\begin{enumerate}
	\item verifies challenge response
	\item fetches the public key for URI (using SSL)
	\item randomly generates a one-way relationship id (REQUESTED-RELID)
	\item randomly generates a one-way request id (REQUESTED-REQID)
	\item encrypts REQUESTED-RELID to friender and signs it
	\item makes message available to be fetched using REQUESTED-REQID to identify it
	\item redirects the user's browser to https://URI/return-relid?uri=FR-URI\&reqid=REQUESTED-REQID
	\end{enumerate}

\item Friender server processes the return-relid request:
	\begin{enumerate}
	\item verifies browser is logged in as owner
	\item fetches public key for FR-URI (using SSL)
	\item fetches encrypted REQUESTED-RELID from FR-URI using REQUESTED-REQID
	\item decrypts and verifies REQUESTED-RELID
	\item randomly generates RETURNED-RELID
	\item randomly generates RETURNED-REQID
	\item encrypts "REQUESTED-RELID RETURNED-RELID" to friendee and signs it
	\item makes message available at to be fetched using RETURNED-REQID to identify it
	\item redirects the friender to https://FR-URI/friend-final?uri=URI\&reqid=REQID
	\end{enumerate}

\item Friendee server processes the friend-final request:
	\begin{enumerate}
	\item fetches encrypted RETURNED-RELID using RETURNED-REQID
	\item decrypts and verifies message, must contain correct REQUESTED-RELID
	\item stores request for friendee to accept/deny
	\end{enumerate}
\end{enumerate}

\section{Friendship Accept/Deny}

The next time the friendee logs in they are presented with the friendship
request. They can either accept or deny the request. 

The acceptor sends accept notification with both relids. They are sent back
indicating that the friendship was registered on the other end. The acceptor
then registers the friendship and sends a registered message. It can can now
send broadcast key and broadcast tree insertion messages. The other end can
send these messages after it receives the registered message.

\section{Publishing to Friends}

\subsection{Broadcast Encryption Keys}

Use a broadcast key. Gives the sender some control over who can read broadcast
messages. Requires sending out session keys ahead of time. This does not stop a
bad friend from forwarding a session key. But the protocol doesn't have this
feature. Session keys must be signed.

\subsection{Distribution Tree}

The most simple approach is to iterate through your list of friends and deliver
broadcast messages. However, a large number of friends multiplied by lots of
activity results in many messages.  

To mitigate this cost a user's set of friends is organized into a complete
binary tree and broadcast messages are forwarded through this tree, resulting
in log n steps to distribute a broadcast.

Doing this means that each of your friends must know the server at which a
couple of your other friends have their identity hosted. This revelation does
not cross friend-partition boundaries, however. It is a minor weakening of the
claim that your friend list is private.

This reduces the time to get messages out, but complicates matters when a node
fails to forward a message on behalf of a friend. The approach I'm considering
is to send acknowledgment messages back up the tree of friends and back to the
originator as confirmation. If the ack does not arrive back in time there is a
distribution problem that needs to be corrected by discovering the faulty node
and removing it or swapping it with a leaf node. The message can then be
rebroadcast. 

A faulty node can be reported to the originator at the point of failure, either
due to a timeout, or an inability to connect. 

\subsection{Tree Generaion Numbers}

To speed up the distribution of messages, a user's list of friends is 
organized into a tree. Each friend forwards broadcast messages on to 
other friends. This requires that there be agreement on the correct 
structure of the tree among a user's friends. Friends do not need to 
have a view of the complete tree, just their children.

Another way to view the issue of agreement is that all operations that 
modify the friend's view of the tree need to be atomic across all 
friends that are involved in the modification. A message that passes 
through the tree must either see all of a change or none of it. When 
adding a member to the tree this is easy, the new member is added as a 
leaf node and just one node (its parent) needs to be modified.

When removing a node from the distribution tree there is more than one 
modification that needs to be made among the friends. To ensure that 
these modifications are atomic, I am using a tree generation number. 
Each broadcast message carries with it the generation of the 
distribution tree that should be used to propagate the message. When the 
tree is modified, the generation number is incremented. Friend nodes 
that are modified keep the old tree generation data around during the 
modification process and continue to use it for any messages that arrive 
with the old generation number. When the changes are complete, new 
messages are sent with the new generation number, making the new tree 
effective.

While this gives us atomic operations, it has the unfortunate property 
that all nodes need to have their generation number updated even if they 
are not involved in an update to the tree. To get around this, friends 
do not look up the exact generation number of messages to find forward 
information, but rather the youngest generation that is less than or 
equal to the generation number of the message. This lets us avoid 
updating the generation number with every friend.

\subsection{Roles and Friendship Types}

The different roles poeple play in their lives are often reflected in a
categorization of their relationships. It would be useful in DSNP to support
friendship categories. Each category should have it's own broadcast key and the
distribution trees should not be disconnected. The categories should not
necessarily be comprised of disjoint sets of friends however. Your brother can
be both in your friend category and your family category. You grandmonther,
however, can just be in your family category so you can shelter her from the
details of that crazy weekend in Montreal.

\subsection{Login as a Friend}

This allows people to login to a friend's page to see the content that has been
summarized by their feed. It is an open-id style login.

User:     URI,    RELID
Friend:   FR-URI, FR-RELID

\begin{enumerate}
\item User fills in a login-as-friend form
	\begin{enumerate}
	\item User visits https://FR-URI/login-as-friend
	\item submits URI
	\end{enumerate}

\item The friend's server processes the request.
	\begin{enumerate}
	\item verifies that URI is a friend
	\item randomly generates a token
	\item encrypts the token to the user and signs it.
	\item makes it available under to be fetched with FR-RELID as identifier
	\item redirects user to https://URI/return-token?uri=FR-URI
	\end{enumerate}

\item  The user's server verifies the user owns the identitiy.
	\begin{enumerate}
	\item checks that FR-URI is a friend
	\item if browser is not logged in, fails the process (possibly with redirect to loging screen)
	\item fetches the token using from FR-URI using FR-RELID to identify it
	\item decrypts and verifies the token
	\item redirects the browser to https://FP-URI/submit-token?uri=URI\&token=TOK
	\end{enumerate}

\item Friend login final
	\begin{enumerate}
	\item Verifies (URI, TOK)
	\item grants friend credentials to the browser using cookie-based session.
	\end{enumerate}
\end{enumerate}

\subsection{Characteristics}

Your list of friends can be treated as private information. It is impossible
for someone to probe your list of friends if you don't want to publish it.
The list must be explictly granted, even to current friends (see next point
for a minor exception). This makes it possible to partition your
relationships into differnet groups, such as family, close friends,
co-workers, and people you hardly know but might like to get to know better.

The user must trust the server that is hosting their identity. But like email,
blogs, and other forms of internet identity, we are free to choose our service
providers and the truely paranoid and technically inclined are free to host
their own identity on their own server. Perhaps a more difficult problem is
that the user must trust the server that is hosting their friend's identities.
However, this is not a problem unique to DSNP. This is true of email as well.

\subsection{Notifications}

\begin{itemize}
\item notification\_broadcast           (user message to a group)
\item notification\_message             (user message direct)
\item notification\_remote\_message     (author-subject message to a group)
\item notification\_remote\_publication (author-subject message to the author)
\end{itemize}

\end{document}
